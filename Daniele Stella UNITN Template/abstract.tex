\chapter*{Abstract}
\label{abtract}
\addcontentsline{toc}{chapter}{Abstract} % add to index

In today's technological landscape, the interplay between hardware constraints and software advancements remains at the forefront of innovation. The domain of wireless communication, an ever-evolving field, exemplifies this dynamic intersection. As devices become increasingly compact and mobile, memory limitations emerge as significant challenges, demanding both inventive hardware solutions and computational strategies to harness their full potential. Amidst this backdrop, the integration and optimization of memory-constrained devices become paramount, heralding a new era of opportunities and complexities.

This dissertation, stemming from my internship at the University of Trento, delves into these intricate challenges with a dual-pronged approach. The first focal point is a comprehensive characterization of the Spark SR1020 Ultra-wideband (UWB) radios. These devices, with their unique attributes and capabilities, offer a microcosm of the broader challenges and solutions within the wireless communication sector. The second strand of investigation is geared towards the adaptation and exploration of Distributed Machine Learning (DML) techniques. Recognizing the constraints posed by limited memory, these DML methods are meticulously tailored to function optimally within such environments.

Drawing from both extensive literature reviews and hands-on experiences during my internship, this research aspires to bridge the gap between theoretical knowledge and practical application, providing insights that are both deep and actionable for the ever-evolving domain of wireless communication.

The cornerstone of this dissertation is anchored in a meticulous examination of the theoretical aspects of wireless communication. By dissecting its intricate layers, the objective is to lay a strong foundation for readers, enabling them to fully grasp the complexities and nuances of this field. This deep dive aims to illuminate the challenges that have historically plagued wireless communication, and more importantly, to spotlight the transformative potential that Distributed Machine Learning (DML) techniques offer as modern-day solutions.

Armed with this foundational knowledge, the narrative then delves into a thorough exploration of the Spark SR1020 UWB radios and the accompanying Family Evaluation Kits. In this segment, a detailed analysis is carried out on the radios' design, architecture, and operational principles. The functionalities of these radios are put under the microscope, revealing insights into how they function. Their performance metrics, gleaned from rigorous testing and experiments, provide empirical evidence of their capabilities. Furthermore, the unique attributes that set these radios apart in the crowded landscape of wireless devices are brought to the fore.

One of the standout features of the Spark SR1020 UWB radios is their low-power design. In an era where energy efficiency is paramount, these radios prove their mettle. Their design philosophy underscores the importance of minimizing energy consumption, making them particularly relevant in today's communication networks. This feature becomes even more critical in environments laden with potential interferences. In such scenarios, the deployment of multiple such low-power devices becomes an invaluable strategy, acting as a countermeasure to environmental challenges and ensuring uninterrupted, robust communication.

Shifting focus to the cutting-edge realm of Distributed Machine Learning (DML), the research embarks on an analytical journey by initially appraising an established self-learning ML pipeline. This step is crucial in understanding the existing methodologies, their strengths, and potential areas for improvement. It provides a clear picture of the current state of the art, offering a baseline against which novel techniques and algorithms can be compared and contrasted.

In response to the challenges posed by memory-constrained devices, this dissertation introduces innovative algorithmic designs meticulously crafted to operate within these bounds. Central to this exploration are three pivotal aggregation algorithms, each presenting its unique approach and strategy. To provide a multifaceted understanding, two of these three algorithms further bifurcate into two specialized variants, broadening the scope of exploration and evaluation.

The rigorous experimental phase of the research plays a crucial role in shedding light on the practical implications of these algorithms. Within this framework, a particular emphasis is placed on a seemingly elementary score-based algorithm. Despite its simplicity, the findings reveal its remarkable efficiency, especially when deployed in constrained environments. Such revelations underscore the notion that complexity doesn't always equate to efficacy. In juxtaposition, the research also delves into the intricacies of certain normalization techniques, highlighting the challenges that arise when implemented, thereby underscoring the importance of careful algorithmic selection based on the specific requirements and constraints of the deployment environment.

In its entirety, this dissertation offers a structured exploration into the nuanced relationship between hardware and computational strategies in the wireless communication domain. By presenting a comprehensive study of the Spark SR1020 radios and weaving these insights seamlessly with advanced DML paradigms, the work stands as a valuable resource for those navigating the complexities of hardware integration and algorithmic advancements in wireless networks.

All tools and software developed during this research, encompassing both the testing protocols for the radios and the algorithmic simulations for ultra-wideband node networks, are made accessible at the following GitHub repository: \url{https://github.com/StellaDaniele/Distributed-ML-UWB-Thesis}.
\newpage



