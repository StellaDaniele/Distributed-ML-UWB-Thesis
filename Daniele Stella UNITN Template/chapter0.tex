\chapter{Introduction}
\label{cha:introduction} 

The dawn of the digital age brought with it a transformative wave of technological advancements, revolutionizing industries and societies across the globe. Within this vast panorama of progress, wireless communication stands out as a linchpin, driving innovations in sectors ranging from healthcare and agriculture to entertainment and defense. As these networks burgeon, so does the proliferation of devices designed to operate within them, many of which face the unique challenge of performing complex tasks within the confines of constrained memory capacities. It is within this context that the present dissertation unfolds, weaving a narrative that juxtaposes the precise hardware characterization with the exciting horizons of computational strategies.

The Spark SR1020 radios, a testament to the strides made in wireless communication hardware, have garnered significant attention. Their design is emblematic of the balance between performance and power conservation — a balance crucial in today's interconnected ecosystems, especially when environmental interference becomes a daunting challenge. My immersion into the detailed characterization of these radios during my tenure at the University of Trento provided insights into their functional nuances, underscoring their role as cornerstones in modern communication infrastructures.

However, while the hardware evolution is undeniable, the quest for optimizing performance goes beyond mere physical components. The computational strategies that underpin these devices, especially in the realm of machine learning, are equally pivotal. The Distributed Machine Learning (DML) approach, tailored for memory-constrained devices, emerges as a beacon of promise. By harnessing the power of DML, we can potentially redefine the paradigms of data processing, analysis, and decision-making, even within the confines of limited memory.

Beginning with a foundational understanding of wireless communication, we'll delve into the heart of the Spark SR1020 radios' characterization. This will set the stage for an exploration of innovative DML paradigms, elucidating their transformative potential in reshaping how memory-constrained devices function.

As we traverse this intricate landscape, the dissertation's intent remains clear — to offer a holistic view, a bridge between the tangible realms of hardware and the abstract, yet profoundly impactful, domains of computational algorithms.

\newpage


