\chapter{Conclusions}
\label{cha:conclusions}

The culmination of three years of intensive study, this dissertation illuminates the multi-faceted challenges and potential solutions in the realm of wireless communication with memory-constrained devices. The research journey traversed two primary, interconnected terrains: a meticulous characterization of the Spark SR1020 radios and the pioneering developments in distributed machine learning (DML) tailored for memory-restricted settings.

Through an exhaustive exploration of the Spark SR1020 radios, this study underscored their pivotal role in modern wireless networks. Their distinct attributes, particularly the low-power characteristics, have shown to be indispensable in environments where managing environmental interferences is paramount. This characterization serves not just as a contribution to the field but also as a beacon for future hardware development and optimization.

Transitioning to DML, this dissertation has provided an in-depth examination of existing self-learning ML pipelines, laying a foundation upon which novel algorithms were introduced. The surprising efficacy of the elementary score-based algorithm, in particular, presents both an exciting avenue for further exploration and a testament to the potential of simplicity in design.

Furthermore, the comparative evaluation of various aggregation algorithms and normalization techniques emphasizes the need for continued innovation. While significant strides were made, the field remains ripe for further research and exploration, especially as technological advancements continue to reshape the landscape of wireless communication.

The harmonious convergence of hardware and computational strategies, as demonstrated through the study of the Spark SR1020 radios and DML paradigms, offers a blueprint for navigating the intricate maze of memory constraints and algorithmic innovation. The challenges faced and the solutions proposed herein not only reflect the state of the art but also provide a vision for the future of wireless communication systems.

\section{Future Work}
This dissertation provides an in-depth exploration of two primary domains: the characterization of the Spark SR1020 Ultra-wideband radios and the design and evaluation of Distributed Machine Learning (DML) algorithms for memory-constrained environments. While the findings and methodologies presented herein are the culmination of extensive research, there remains room for further refinement and exploration in both domains.


\begin{enumerate}
    \item Characterization of the Spark SR1020 UWB Radios: The power consumption measurements for the Spark SR1020 UWB radios presented in this dissertation were foundational, yet there's scope for refinement. Given the instrumental constraints during my internship at the University of Trento, the assessments were inherently limited. Future endeavors can focus on employing more sophisticated measurement tools and methodologies to provide a more granular understanding of the power dynamics of these radios.
    \item Distributed Machine Learning Algorithms: The DML algorithms developed and evaluated in this work, while innovative, open avenues for continued research. A particular point of interest is the normalization process integrated into one of the algorithms. Theoretically, this process should bolster the algorithm's performance, positioning it as a superior choice among its counterparts. However, the current implementation doesn't harness its full potential. A more nuanced approach, potentially involving iterative refinements and extensive testing, might be instrumental in optimizing its performance. Given more time and resources, it's plausible to posit that this normalization-centric algorithm can outshine the others presented in this research.
\end{enumerate}


In conclusion, while this dissertation marks a significant step forward in bridging the intricacies of hardware constraints with algorithmic advancements in the realm of wireless communication, the journey is far from over. The insights gained from this research pave the way for deeper explorations, promising further innovations in the confluence of UWB radios and DML paradigms.


\newpage




